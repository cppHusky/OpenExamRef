\documentclass[twoside]{ctexart}
\usepackage{bookmark}
\usepackage{fancyhdr}
\usepackage[a4paper,landscape,left=3mm,right=3mm,top=4mm,bottom=12mm]{geometry}
\usepackage{hyperref}
\usepackage{pdfpages}
\usepackage{titletoc}
\newcommand{\includepdfwithpage}[3]{
	\clearpage
	\phantomsection
	\addcontentsline{toc}{section}{#2}
	\includepdf[pagecommand={\thispagestyle{plain}},nup=3x3,pages={#1}]{#3}
}
\title{\textbf{软件工程课件}\\(开卷考试用)}
\author{\textit{cppHusky}}
\date{\today}
\titlecontents{section}[1em]
	{}
	{\bfseries\contentslabel[\thecontentslabel.0]{2em}\MakeUppercase}
	{\bfseries}
	{\titlerule*[.5em]{.}\contentspage}
\begin{document}
\maketitle
\hypersetup{hidelinks}
\fancypagestyle{plain}{
	\fancyhf{}
	\fancyfoot[LE,RO]{\Large\textbf{\thepage}}
}
\pagestyle{plain}
\tableofcontents\newpage
\includepdfwithpage{1-9,11-19}{软件工程概述}{assets/02-软件工程概述}
\includepdfwithpage{1-53}{软件生命周期模型}{assets/03-软件生命周期模型}
\includepdfwithpage{1-3,5-24}{软件需求分析}{assets/04-软件需求分析}
\includepdfwithpage{1-59}{面向对象需求分析}{assets/05-面向对象需求分析}
\includepdfwithpage{1-67}{软件设计}{assets/06-软件设计}
\includepdfwithpage{1-42}{面向对象设计方法}{assets/07-面向对象设计方法}
\includepdfwithpage{1-7}{软件实现}{assets/09-软件实现}
\includepdfwithpage{1-49}{软件测试}{assets/10-软件测试}
\includepdfwithpage{1-25}{剩余章节}{assets/11-剩余章节}
\end{document}
